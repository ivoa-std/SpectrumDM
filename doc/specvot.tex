
\section{VOTABLE serialization}

\subsection{Mapping Schema to VOTABLE}

We reproduce below the XML schema instance example as a VOTABLE instance example.
To go from the XML instance to the VOTABLE
instance, we:
\begin{itemize}
\item  - map the top level element to a RESOURCE
\item  - map all elements with simple content to PARAM
\item  - map all elements with complex content to GROUP
\item  - map the element names (with appropriate path) to values of the
utype attribute,
\item  - but, handle the FIELDS and Data elements in a special way.
The FIELDS element is used to define the table fields and the Data
element is used to define the table data. 
\item  - but, also,  all the second level elements below RESOURCE
except SPECTRUM map to an initial TABLE, while we map SPECTRUM 
to a second TABLE.
\item  - most of the elements extend the Param element, to which I have added
an optional name attribute that I have not used in the instance.
If this attribute is used, it can hold the name attributes of the PARAM and
FIELD; otherwise the relevant attributes could be filled with the
same value as the utype (without namespace prefix).

\end{itemize}

How can this be generalized to mapping an arbitrary data model
schema to VOTABLE? The only tricky parts are 
\begin{itemize}
\item  {\bf Spotting where the tabledata parts are. } 
We could require any DM schema that maps to VOTABLE
to include elements called FIELDS and Data (perhaps ROWS would be 
a better name), otherwise you would get a VOTABLE with no data section.
\item  {\bf Spotting where to start the main TABLE (i.e. the fact that
SPECTRUM is special). }  We could change the schema to
have an explicit attribute, annotation or other marker to tell us this.
\end{itemize}

These issues will require further discussion for future models.

\subsection{A VOTABLE instance}

The VOTable version of Spectrum uses a single VOTable {\lcaret}TABLE{\rcaret}
(Note that this may appear as one of many tables within an SED VOTable).
The data model fields described above as arrays map to
VOTable FIELDs, while the remaining fields map to PARAM.

We use nested GROUP constructs to delimit data model objects within the
main object, and PARAM and FIELD tags for attributes.
The nesting beyond a single GROUP is optional, as for cases for which
the utypes are unique within a group, the utypes can be used to infer
the datamodel structure. See
http://webtest.aoc.nrao.edu/ivoa-dal for a service returning VOTABLE
Spectrum instances with only one level of GROUP.

Names of fields and parameters are left to the data provider.
The utype and ucd attributes are used to denote data model and UCD tags.
The schema and namespace for the utypes is the XML schema given in section 8.4.
We have made up arbitrary NAME attributes for the PARAM and these
are not to be considered standard; the name fields are free
to be whatever the data provider wants, allowing compatibility with
local archive nomenclature. The NAME attributes for the FIELD elements
are also not standardized (of course they must be the same as in the
matching FIELDrefs); it is the utype attribute which is standardized.

The one departure from the XML schema below is that the `Data'
element and the individual `Point' elements are implicitly represented
by the table structure itself. Perhaps a UTYPE attribute to the
TABLEDATA element could be used to make this explicit.

The examples below describe a single SPECTRUM.


{ \footnotesize
\begin{flushleft}


\begin{fmpage}

\begin{verbatim}
<?xml version="1.0" encoding="UTF-8"?>
<VOTABLE version="1.1"
  xmlns:xsi="http://www.w3.org/2001/XMLSchema-instance"
  xsi:noNamespaceSchemaLocation="xmlns:http://www.ivoa.net/xml/VOTable/VOTable-1.1.xsd" 
  xmlns:spec="http://www.ivoa.net/xml/SpectrumModel/v1.01"
  xmlns="http://www.ivoa.net/xml/VOTable/v1.1">
<RESOURCE utype="spec:Spectrum">

<TABLE utype="spec:Spectrum">   
<GROUP utype="spec:Target">
 <PARAM name="Target" utype="spec:Target.Name" datatype="char" arraysize="*" value="Arp 220"/>
 <PARAM name="TargetPos" utype="spec:Target.pos" unit="deg" datatype="double" 
                                 arraysize="2" value="233.737917 23.503330"/>
 <PARAM name="z" utype="spec:Target.redshift" datatype="float" value="0.0018"/>
</GROUP>

<!-- SegmentType can be Photometry, TimeSeries or Spectrum -->
<PARAM name="Segtype" utype="spec:SegmentType" datatype="char" arraysize="*" 
                                                  value="Photometry" ucd="meta.code"/>
<GROUP name="CoordSys" utype="spec:CoordSys">
 <GROUP utype="CoordSys.SpaceFrame">
   <PARAM name="System" utype="spec:CoordSys.SpaceFrame.Name" ucd="pos.frame" 
                                         datatype="char" arraysize="*" value="ICRS"/>
   <PARAM name="Equinox" utype="spec:CoordSys.SpaceFrame.Equinox" ucd="time.equinox;pos.eq" 
                                         datatype="float" value="2000.0" />
 </GROUP>
 <GROUP utype="spec:CoordSys.TimeFrame">
  <PARAM name="TimeFrame" utype="spec:CoordSys.TimeFrame.Name" ucd="time.scale" datatype="char" 
                                       arraysize="*" value="UTC"/>
 </GROUP>
 <GROUP utype="spec:CoordSys.SpectralFrame">
  <PARAM name="SpectralFrame" utype="spec:CoordSys.SpectralFrame.RefPos" ucd="sdm:spect.frame" 
                                      datatype="char" arraysize="*" value="BARYCENTER"/>
 </GROUP>
</GROUP>

\end{verbatim}
\end{fmpage}

\begin{fmpage}
\begin{verbatim}


<GROUP utype="spec:Char">
 <GROUP utype="spec:Char.SpatialAxis">
   <PARAM name="SpatialAxisName" utype="name" ucd="pos.eq" unit="deg" value="Sky"/>
   <GROUP utype="spec:Char.SpatialAxis.Coverage">
    <GROUP utype="spec:Char.SpatialAxis.Coverage.Location">
     <PARAM name="SkyPos" utype="Char.SpatialAxis.Coverage.Location.Value" 
                                   ucd="pos.eq" unit="deg" 
                                   datatype="double" arraysize="2" value="132.4210 12.1232"/>
   </GROUP>
   <GROUP utype="Bounds">
     <PARAM name="SkyExtent" utype="Char.SpatialAxis.Coverage.Extent" ucd="pos.angDistance;instr.fov" 
                     datatype="double" unit="arcsec" value="20"/>
   </GROUP>
  </GROUP>
 </GROUP>



 <GROUP utype="spec:Char.TimeAxis">
  <PARAM name="TimeAxisName" utype="Char.TimeAxis.Name" ucd="time" unit="d" value="Time"/>
  <GROUP utype="Char.TimeAxis.Coverage">
   <GROUP utype="Char.TimeAxis.Coverage.Location">
    <PARAM name="TimeObs" utype="Char.TimeAxis.Coverage.Location.Value" ucd="time.epoch;obs"
                                   datatype="double" value="52148.3252"/>
   </GROUP>
   <GROUP utype="Char.TimeAxis.Coverage.Bounds">
    <PARAM name="TimeExtent" utype="Char.TimeAxis.Coverage.Bounds.Extent" ucd="time.duration"
                     unit="s" datatype="double" value="1500.0" />
    <PARAM name="TimeStart" utype="Char.TimeAxis.Coverage.Bounds.Start" ucd="time.start" unit="s" 
                                     datatype="double" value="52100.000" />
    <PARAM name="TimeStop" utype="Char.TimeAxis.Coverage.Bounds.Stop" ucd="time.end" unit="s" 
                                     datatype="double" value="52300.000" />
   </GROUP>
   <GROUP utype="Char.TimeAxis.Coverage.Support">
    <PARAM name="TimeExtent" utype="Char.TimeAxis.Coverage.Support.Extent" ucd="time.duration;obs.exposure"
                     unit="s" datatype="double" value="1500.0" />
    <PARAM name="TimeStart" utype="Char.TimeAxis.Coverage.Bounds.Start" ucd="time.start" unit="s" 
                                     datatype="double" value="52100.000" />
    <PARAM name="TimeStop" utype="Char.TimeAxis.Coverage.Bounds.Stop" ucd="time.end" unit="s" 
                                     datatype="double" value="52300.000" />
   </GROUP>

  </GROUP>
 </GROUP>

 <GROUP utype="spec:Char.SpectralAxis">
  <PARAM name="SpectralAxisName" utype="Char.SpectralAxis.Name" ucd="em.wl" unit="angstrom" value="Wavelength"/>
  <GROUP utype="Char.SpectralAxis.Coverage">
   <GROUP utype="Char.SpectralAxis.Coverage.Bounds">
    <PARAM name="SpectralExtent" utype="Char.SpectralAxis.Coverage.Bounds.Extent" ucd="instr.bandwidth" 
                     unit="angstrom" datatype="double" value="3000.0"/>
   </GROUP>
  </GROUP>
 </GROUP>
</GROUP>
\end{verbatim}
\end{fmpage}

\begin{fmpage}
\begin{verbatim}

<GROUP utype="spec:Curation">
 <PARAM name="Publisher" utype="spec:Curation.Publisher" ucd="meta.curation" 
                       datatype="char" arraysize="*" value="SAO"/>
 <PARAM name="PubID" utype="spec:Curation.PublisherID" ucd="meta.ref.url;meta.curation" datatype="char" 
                                       arraysize="*" value="ivo://cfa.harvard.edu"/>
 <PARAM name="Contact" utype="spec:Curation.Contact.Name" ucd="meta.bib.author;meta.curation" 
                       datatype="char" arraysize="*" value="Jonathan McDowell"/>
 <PARAM name="email" utype="spec:Curation.Contact.Email" ucd="meta.email" datatype="char" 
                                      arraysize="*" value="jcm@cfa.harvard.edu"/>
</GROUP>


<GROUP utype="spec:DataID">
 <PARAM name="Title" utype="spec:DataID.Title" datatype="char" arraysize="*" value="Arp 220 SED"/>
 <PARAM name="Creator" utype="spec:Segment.DataID.Creator" ucd="meta.curation" datatype="char" 
                                      arraysize="*" value="ivo://sao/FLWO"/>
 <PARAM name="DataDate" utype="spec:DataID.Date" ucd="time.epoch;meta.dataset"
                      datatype="char" arraysize="*" value="2003-12-31T14:00:02Z"/>
 <PARAM name="Version" utype="spec:DataID.Version" ucd="meta.version;meta.dataset"
                      datatype="char" arraysize="*" value="1"/>
 <PARAM name="Instrument" utype="spec:DataID.Instrument" ucd="meta.id;instr" datatype="char" 
                                    arraysize="*" value="BCS"/>
 <PARAM name="Filter" utype="spec:DataID.Collection" ucd="inst.filter.id" datatype="char" 
                                   arraysize="*" value="G300"/>
 <PARAM name="CreationType" utype="spec:DataID.CreationType" datatype="char" arraysize="*" value="Archival"/>
 <PARAM name="Logo" utype="spec:DataID.Logo" ucd="meta.ref.url" datatype="char" 
                                        arraysize="*" value="http://cfa-www.harvard.edu/nvo/cfalogo.jpg"/>
</GROUP>

<GROUP utype="spec:Derived">
 <PARAM name="SNR" utype="spec:Derived.SNR" datatype="float" value="3.0"/>
</GROUP>

<GROUP utype="spec:Data">

<GROUP utype="spec:Data.SpectralAxis">
 <FIELDref ref="Coord"/>

 <GROUP utype="spec:Data.SpectralAxis.Accuracy">
  <FIELDref ref="BinLow"/>
  <FIELDref ref="BinHigh"/>
 </GROUP>
<!-- In this case Resolution is demoted from Field to Param since it is constant -->
 <PARAM name="Resolution" utype="spec:Data.SpectralAxis.Resolution" 
      ucd="spect.resolution;em.wl"    unit="angstrom" datatype="float" value="14.2"/>
</GROUP>

\end{verbatim}
\end{fmpage}

\begin{fmpage}
\begin{verbatim}
<GROUP utype="spec:Data.FluxAxis">
 <FIELDref ref="Flux1"/>
 <GROUP utype="spec:Data.FluxAxis.Accuracy">
  <FIELDref ref="ErrorLow"/>
  <FIELDref ref="ErrorHigh"/>
  <PARAM name="SysErr" utype="SysErr" unit="" datatype="float" value="0.05"/>
 </GROUP>
 <FIELDref ref="Quality"/>
</GROUP>
</GROUP>

<FIELD name="Coord" ID="Coord" utype="spec:Data.SpectralAxis.Value" ucd="em.wl"
              datatype="double" unit="angstrom"/>
<FIELD name="BinLow" ID="BinLow" utype="spec:Data.SpectralAxis.BinLow" 
              ucd="em.wl;stat.min"
              datatype="double" unit="angstrom"/>
<FIELD name="BinHigh" ID="BinHigh" utype="spec:Data.SpectralAxis.BinHigh"
               ucd="em.wl;stat.max"
              datatype="double" unit="angstrom"/>
<FIELD name="Flux" ID="Flux1" utype="spec:Data.FluxAxis.value" ucd="phot.flux.density;em.wl"
              datatype="double" unit="erg cm**(-2) s**(-1) angstrom**(-1)"/>
<FIELD name="ErrorLow" ID="ErrorLow" utype="spec:Data.FluxAxis.Accuracy.StatErrLow" 
              datatype="double" unit="erg cm**(-2) s**(-1) angstrom**(-1)"/>
<FIELD name="ErrorHigh" ID="ErrorHigh" utype="spec:Data.FluxAxis.Accuracy.StatErrHigh" 
              datatype="double" unit="erg cm**(-2) s**(-1) angstrom**(-1)"/>
<FIELD name="Quality" ID="Quality" datatype="int" utype="spec:Data.FluxAxis.Quality"/>
<DATA>
<TABLEDATA>
<!-- Note slightly nonlinear wavelength solution -->
<!-- Second row is upper limit -->
<!-- Third row has quality mask set -->
<TR><TD>3200.0</TD><TD>3195.0</TD><TD>3205.0</TD><TD>1.38E-12</TD><TD>5.2E-14</TD><TD>6.2E-14</TD>
                                                                                    <TD>0</TD></TR>
<TR><TD>3210.5</TD><TD>3205.0</TD><TD>3216.0</TD><TD>1.12E-12</TD><TD>1.12E-12</TD>
                                                                     <TD>0</TD><TD>0</TD></TR>
<TR><TD>3222.0</TD><TD>3216.0</TD><TD>3228.0</TD><TD>1.42E-12</TD><TD>1.3E-14</TD>
                                                                     <TD>0.2E-14</TD><TD>3</TD></TR>
</TABLEDATA>
</DATA>
</TABLE>
</RESOURCE>
</VOTABLE>

\end{verbatim}
\end{fmpage}

\end{flushleft}
}


A second example, based on the reference SSAP proxy service for
the JHU SDSS spectrum archive:

\begin{landscape}
{\scriptsize
\begin{flushleft}
\begin{fmlpage}

\begin{verbatim}
<?xml version="1.0" encoding="UTF-8"?>
<VOTABLE xmlns:xsi="http://www.w3.org/2001/XMLSchema-instance"
   xsi:noNamespaceSchemaLocation="xmlns:http://www.ivoa.net/xml/VOTable/VOTable-1.1.xsd" 
   xmlns:spec="http:www.ivoa.net/xml/SpectrumModel/v1.0" version="1.1">
<RESOURCE utype="spec:Spectrum">
<DESCRIPTION>Spectrum dataset generated by DALServer</DESCRIPTION>
<TABLE utype="spec:Spectrum">
<FIELD ID="DataSpectralValue" name="DataSpectralValue" datatype="double" ucd="em.wl" utype="spec:Spectrum.Data.SpectralAxis.Value" unit="angstrom">
<DESCRIPTION>Spectral coordinates for points</DESCRIPTION>
</FIELD>
<FIELD ID="DataSpectralBinLow" name="DataSpectralBinLow" datatype="double" ucd="em.wl;stat.min" utype="spec:Spectrum.Data.SpectralAxis.Accuracy.BinLow" unit="angstrom">
<DESCRIPTION>Spectral coord bin lower end</DESCRIPTION>
</FIELD>
<FIELD ID="DataSpectralBinHigh" name="DataSpectralBinHigh" datatype="double" ucd="em.wl;stat.max" utype="spec:Spectrum.Data.SpectralAxis.Accuracy.BinHigh" unit="angstrom">
<DESCRIPTION>Spectral coord bin upper end</DESCRIPTION>
</FIELD>
<FIELD ID="DataFluxValue" name="DataFluxValue" datatype="double" ucd="phot.flux.density;em.wl" utype="spec:Spectrum.Data.FluxAxis.Value" 
   unit="10**(-17) erg/cm**2/s/angstrom">
<DESCRIPTION>Flux values for points</DESCRIPTION>
</FIELD>
<FIELD ID="DataFluxStatErrLow" name="DataFluxStatErrLow" datatype="double" ucd="phot.flux.density;em.wl" utype="spec:Spectrum.Data.FluxAxis.Accuracy.StatErrLow"
   unit="10**(-17) erg/cm**2/s/angstrom"><DESCRIPTION>Flux lower error</DESCRIPTION>
</FIELD>
<FIELD ID="DataFluxStatErrHigh" name="DataFluxStatErrHigh" datatype="double" ucd="phot.flux.density;em.wl" utype="spec:Spectrum.Data.FluxAxis.Accuracy.StatErrHigh"
 unit="10**(-17) erg/cm**2/s/angstrom"> <DESCRIPTION>Flux upper error</DESCRIPTION>
</FIELD>
<FIELD ID="DataFluxQuality" name="DataFluxQuality" datatype="long" ucd="meta.code.qual;phot.flux" utype="spec:Spectrum.Data.FluxAxis.Quality">
<DESCRIPTION>Flux measurement quality mask</DESCRIPTION>
</FIELD>
<GROUP ID="Spectrum" name="Spectrum" utype="spec:Spectrum">
<DESCRIPTION>General Dataset Metadata</DESCRIPTION>
<PARAM ID="DataModel" datatype="char" name="DataModel" utype="spec:Spectrum.DataModel" value="Spectrum 1.0" arraysize="*">
<DESCRIPTION>Datamodel name and version</DESCRIPTION>
</PARAM>
<PARAM ID="DatasetType" datatype="char" name="DatasetType" utype="spec:Spectrum.Type" value="Spectrum" arraysize="*">
<DESCRIPTION>Dataset or segment type</DESCRIPTION>
</PARAM>
<PARAM ID="DataLength" datatype="long" name="DataLength" utype="spec:Spectrum.Length" value="3820">
<DESCRIPTION>Number of points</DESCRIPTION>
</PARAM>
</GROUP>
<GROUP ID="DataID" name="DataID" utype="spec:DataID">
<DESCRIPTION>Dataset Identification Metadata</DESCRIPTION>
<PARAM ID="Title" datatype="char" name="Title" ucd="meta.title;meta.dataset" utype="spec:Spectrum.DataID.Title" 
 value="SDSS J142906.30+016000.00 Rosat_d 0535-51998-01" arraysize="*"><DESCRIPTION>Dataset Title</DESCRIPTION>
</PARAM>
<PARAM ID="Creator" datatype="char" name="Creator" utype="spec:Spectrum.DataID.Creator" value="sdss" arraysize="*">
<DESCRIPTION>Dataset creator</DESCRIPTION>
</PARAM>
<PARAM ID="Collection" datatype="char" name="Collection" utype="spec:Spectrum.DataID.Collection" value="ivo://sdss/dr5/spec" arraysize="*">
<DESCRIPTION>Data collection to which dataset belongs</DESCRIPTION>
</PARAM>
\end{verbatim}
\end{fmlpage}

\begin{fmlpage}
\begin{verbatim}
<PARAM ID="CreatorDID" datatype="char" name="CreatorDID" ucd="meta.id" utype="spec:Spectrum.DataID.CreatorDID" value="ivo://sdss/dr5/spec#150812447593201664"
 arraysize="*"><DESCRIPTION>Creator's ID for the dataset</DESCRIPTION>
</PARAM>
<PARAM ID="CreatorDate" datatype="char" name="CreatorDate" ucd="time;meta.dataset" utype="spec:Spectrum.DataID.Date" value="2001-03-31T09:10:18.8600000-04:00" 
 arraysize="*"><DESCRIPTION>Data processing/creation date</DESCRIPTION>
</PARAM>
<PARAM ID="CreatorVersion" datatype="char" name="CreatorVersion" ucd="meta.version;meta.dataset" utype="spec:Spectrum.DataID.Version" value="3.36.10" 
 arraysize="*"><DESCRIPTION>Version of dataset</DESCRIPTION>
</PARAM>
<PARAM ID="Instrument" datatype="char" name="Instrument" ucd="meta.id;instr" utype="spec:Spectrum.DataID.Instrument" value="SDSS 2.5-M SPEC1 v4_7" arraysize="*">
<DESCRIPTION>Instrument name</DESCRIPTION>
</PARAM>
<PARAM ID="DataSource" datatype="char" name="DataSource" utype="spec:Spectrum.DataID.DataSource" value="survey" arraysize="*">
<DESCRIPTION>Original source of the data</DESCRIPTION>
</PARAM>
<PARAM ID="CreationType" datatype="char" name="CreationType" utype="spec:Spectrum.DataID.CreationType" value="Archival" arraysize="*">
<DESCRIPTION>Dataset creation type</DESCRIPTION>
</PARAM>
</GROUP>
<GROUP ID="Curation" name="Curation" utype="spec:Curation">
<DESCRIPTION>Curation Metadata</DESCRIPTION>
<PARAM ID="Publisher" datatype="char" name="Publisher" ucd="meta.curation" utype="spec:Spectrum.Curation.Publisher" value="DALServer Proxy" arraysize="*">
<DESCRIPTION>Dataset publisher</DESCRIPTION>
</PARAM>
<PARAM ID="PublisherDID" datatype="char" name="PublisherDID" ucd="meta.ref.url;meta.curation" utype="spec:Spectrum.Curation.PublisherDID" 
 value="ivo://jhu/sdss/dr5#150812447593201664" arraysize="*"><DESCRIPTION>Publisher's ID for the dataset ID</DESCRIPTION>
</PARAM>
<PARAM ID="Rights" datatype="char" name="Rights" utype="spec:Spectrum.Curation.Rights" value="public" arraysize="*">
<DESCRIPTION>Restrictions on data access</DESCRIPTION>
</PARAM>
</GROUP>
<GROUP ID="Target" name="Target" utype="spec:Target">
<DESCRIPTION>Target Metadata</DESCRIPTION>
<PARAM ID="TargetName" datatype="char" name="TargetName" ucd="meta.id;src" utype="spec:Spectrum.Target.Name" value="SDSS J142906.30+016000.00" arraysize="*">
<DESCRIPTION>Target name</DESCRIPTION>
</PARAM>
<PARAM ID="TargetDescription" datatype="char" name="TargetDescription" utype="spec:Spectrum.Target.Description" value="0535-51998-01" arraysize="*">
<DESCRIPTION>Target description</DESCRIPTION>
</PARAM>
<PARAM ID="TargetClass" datatype="char" name="TargetClass" ucd="src.class" utype="spec:Spectrum.Target.Class" value="Rosat_d" arraysize="*">
<DESCRIPTION>Object class of observed target</DESCRIPTION>
</PARAM>
<PARAM ID="TargetPos" unit="deg" datatype="double" name="TargetPos" ucd="pos.eq;src" utype="spec:Spectrum.Target.Pos" value="217.276250 1.289608" arraysize="2">
<DESCRIPTION>Target RA and Dec</DESCRIPTION>
</PARAM>
<PARAM ID="SpectralClass" datatype="char" name="SpectralClass" ucd="src.spType" utype="spec:Spectrum.Target.SpectralClass" value="Star" arraysize="*">
<DESCRIPTION>Object spectral class</DESCRIPTION>
</PARAM>
<PARAM ID="Redshift" datatype="double" name="Redshift" ucd="src.redshift" utype="spec:Spectrum.Target.Redshift" value="0.000197439">
<DESCRIPTION>Target redshift</DESCRIPTION>
</PARAM>
<PARAM ID="VarAmpl" datatype="float" name="VarAmpl" ucd="src.var.amplitude" utype="spec:Spectrum.Target.VarAmpl" value="0">
<DESCRIPTION>Target variability amplitude (typical)</DESCRIPTION>
</PARAM>
</GROUP>
\end{verbatim}
\end{fmlpage}

\begin{fmlpage}
\begin{verbatim}
<GROUP ID="Derived" name="Derived" utype="spec:Derived">
<DESCRIPTION>Derived Metadata</DESCRIPTION>
<PARAM ID="DerivedRedshift" datatype="double" name="DerivedRedshift" utype="spec:Spectrum.Derived.Redshift.Value" value="0.000197439">
<DESCRIPTION>Measured redshift for spectrum</DESCRIPTION>
</PARAM>
<PARAM ID="RedshiftStatError" datatype="float" name="RedshiftStatError" ucd="stat.error;src.redshift" utype="spec:Spectrum.Derived.Redshift.StatError"
   value="0.000166012"><DESCRIPTION>Error on measured redshift</DESCRIPTION>
</PARAM>
<PARAM ID="RedshiftConfidence" datatype="float" name="RedshiftConfidence" utype="spec:Spectrum.Derived.Redshift.Confidence" value="0.360793">
<DESCRIPTION>Confidence value on redshift</DESCRIPTION>
</PARAM>
<PARAM ID="DerivedVarAmpl" datatype="float" name="DerivedVarAmpl" ucd="src.var.amplitude;arith.ratio" utype="spec:Spectrum.Derived.VarAmpl" value="0">
<DESCRIPTION>Variability amplitude as fraction of mean</DESCRIPTION>
</PARAM>
</GROUP>
<GROUP ID="CoordSys" name="CoordSys" utype="spec:CoordSys">
<DESCRIPTION>Coordinate System Metadata</DESCRIPTION>
<PARAM ID="SpaceFrameName" datatype="char" name="SpaceFrameName" utype="spec:Spectrum.CoordSys.SpaceFrame.Name" value="FK5" arraysize="*">
<DESCRIPTION>Spatial coordinate frame name</DESCRIPTION>
</PARAM>
<PARAM ID="SpaceFrameUcd" datatype="char" name="SpaceFrameUcd" utype="spec:Spectrum.CoordSys.SpaceFrame.Ucd" value="pos.eq" arraysize="*">
<DESCRIPTION>Space frame UCD</DESCRIPTION>
</PARAM>
<PARAM ID="SpaceFrameEquinox" unit="yr" datatype="double" name="SpaceFrameEquinox" ucd="time.equinox;pos.frame"
 utype="spec:Spectrum.CoordSys.SpaceFrame.Equinox" value="2000"><DESCRIPTION>Equinox</DESCRIPTION>
</PARAM>
<PARAM ID="TimeFrameName" datatype="char" name="TimeFrameName" ucd="time.scale" utype="spec:Spectrum.CoordSys.TimeFrame.Name" value="TAI" arraysize="*">
<DESCRIPTION>Timescale</DESCRIPTION>
</PARAM>
<PARAM ID="TimeFrameZero" unit="d" datatype="double" name="TimeFrameZero" ucd="time;arith.zp" utype="spec:Spectrum.CoordSys.TimeFrame.Zero" value="0">
<DESCRIPTION>Zero point of timescale in MJD</DESCRIPTION>
</PARAM>
<PARAM ID="TimeFrameRefPos" datatype="char" name="TimeFrameRefPos" ucd="sdm:time.frame" utype="spec:Spectrum.CoordSys.TimeFrame.RefPos" value="Topocentric" 
 arraysize="*"><DESCRIPTION>Location for times of photon arrival</DESCRIPTION>
</PARAM>
<PARAM ID="SpectralFrameUcd" datatype="char" name="SpectralFrameUcd" utype="spec:Spectrum.CoordSys.SpectralFrame.Ucd" value="em.wl" arraysize="*">
<DESCRIPTION>Spectral frame UCD</DESCRIPTION>
</PARAM>
<PARAM ID="SpectralFrameRefPos" datatype="char" name="SpectralFrameRefPos" utype="spec:Spectrum.CoordSys.SpectralFrame.RefPos" value="Topocentric" arraysize="*">
<DESCRIPTION>Spectral frame origin</DESCRIPTION>
</PARAM>
</GROUP>
\end{verbatim}
\end{fmlpage}

\begin{fmlpage}
\begin{verbatim}
<GROUP ID="Char.SpatialAxis" name="Char.SpatialAxis" utype="spec:Char.SpatialAxis">
<DESCRIPTION>Spatial Axis Characterization</DESCRIPTION>
<PARAM ID="SpatialAxisName" datatype="char" name="SpatialAxisName" utype="spec:Spectrum.Char.SpatialAxis.Name" value="Sky" arraysize="*">
<DESCRIPTION>Name for spatial axis</DESCRIPTION>
</PARAM>
<PARAM ID="SpatialAxisUcd" datatype="char" name="SpatialAxisUcd" utype="spec:Spectrum.Char.SpatialAxis.Ucd" value="pos.eq" arraysize="*">
<DESCRIPTION>UCD for spatial coord</DESCRIPTION>
</PARAM>
<PARAM ID="SpatialAxisUnit" datatype="char" name="SpatialAxisUnit" utype="spec:Spectrum.Char.SpatialAxis.Unit" value="deg" arraysize="*">
<DESCRIPTION>Unit for spatial coord</DESCRIPTION>
</PARAM>
<PARAM ID="SpatialLocation" unit="deg" datatype="double" name="SpatialLocation" ucd="pos.eq" utype="spec:Spectrum.Char.SpatialAxis.Coverage.Location.Value" 
  value="217.276250 1.289608" arraysize="2"><DESCRIPTION>Spatial Position</DESCRIPTION>
</PARAM>
<PARAM ID="SpatialExtent" unit="deg" datatype="double" name="SpatialExtent" ucd="instr.fov" utype="spec:Spectrum.Char.SpatialAxis.Coverage.Bounds.Extent"
  value="0.00083333333333333339"><DESCRIPTION>Aperture angular size</DESCRIPTION>
</PARAM>
<PARAM ID="SpatialCalibration" datatype="char" name="SpatialCalibration" ucd="meta.code.qual" utype="spec:Spectrum.Char.SpatialAxis.Accuracy.Calibration" 
 value="calibrated" arraysize="*"><DESCRIPTION>Type of spatial coord calibration</DESCRIPTION>
</PARAM>
</GROUP>
<GROUP ID="Char.SpectralAxis" name="Char.SpectralAxis" utype="spec:Char.SpectralAxis">
<DESCRIPTION>Spectral Axis Characterization</DESCRIPTION>
<PARAM ID="SpectralAxisName" datatype="char" name="SpectralAxisName" utype="spec:Spectrum.Char.SpectralAxis.Name" value="SpectralCoord" arraysize="*">
<DESCRIPTION>Name for spectral axis</DESCRIPTION>
</PARAM>
<PARAM ID="SpectralAxisUcd" datatype="char" name="SpectralAxisUcd" utype="spec:Spectrum.Char.SpectralAxis.Ucd" value="em.wl" arraysize="*">
<DESCRIPTION>UCD for spectral coord</DESCRIPTION>
</PARAM>
<PARAM ID="SpectralAxisUnit" datatype="char" name="SpectralAxisUnit" utype="spec:Spectrum.Char.SpectralAxis.Unit" value="angstrom" arraysize="*">
<DESCRIPTION>Unit for spectral coord</DESCRIPTION>
</PARAM>
<PARAM ID="SpectralLocation" datatype="double" name="SpectralLocation" ucd="instr.bandpass" utype="spec:Spectrum.Char.SpectralAxis.Coverage.Location.Value" 
  value="6515.408759029946"><DESCRIPTION>Spectral coord value</DESCRIPTION>
</PARAM>
<PARAM ID="SpectralExtent" datatype="double" name="SpectralExtent" ucd="instr.bandwidth" utype="spec:Spectrum.Char.SpectralAxis.Coverage.Bounds.Extent" 
 value="5386.6535232062424"><DESCRIPTION>Width of spectrum</DESCRIPTION>
</PARAM>
<PARAM ID="SpectralStart" datatype="double" name="SpectralStart" ucd="em.wl;stat.min" utype="spec:Spectrum.Char.SpectralAxis.Coverage.Bounds.Start"
 value="3822.0819974268243"><DESCRIPTION>Start in spectral coordinate</DESCRIPTION>
</PARAM>
<PARAM ID="SpectralStop" datatype="double" name="SpectralStop" ucd="em.wl;stat.max" utype="spec:Spectrum.Char.SpectralAxis.Coverage.Bounds.Stop" 
 value="9208.7355206330667"><DESCRIPTION>Stop in spectral coordinate</DESCRIPTION>
</PARAM>
<PARAM ID="SpectralBinSize" datatype="double" name="SpectralBinSize" ucd="spect.binSize" utype="spec:Spectrum.Char.SpectralAxis.Accuracy.BinSize" value="0">
<DESCRIPTION>Spectral coord bin size</DESCRIPTION>
</PARAM>
<PARAM ID="SpectralStatError" datatype="double" name="SpectralStatError" ucd="stat.error;em.wl" utype="spec:Spectrum.Char.SpectralAxis.Accuracy.StatError" value="0">
<DESCRIPTION>Spectral coord statistical error</DESCRIPTION>
</PARAM>
\end{verbatim}
\end{fmlpage}

\begin{fmlpage}
\begin{verbatim}

<PARAM ID="SpectralSysError" datatype="double" name="SpectralSysError" ucd="stat.error;em.wl" utype="spec:Spectrum.Char.SpectralAxis.Accuracy.SysError" value="0">
<DESCRIPTION>Spectral coord systematic error</DESCRIPTION>
</PARAM>
<PARAM ID="SpectralCalibration" datatype="char" name="SpectralCalibration" ucd="meta.code.qual" utype="spec:Spectrum.Char.SpectralAxis.Accuracy.Calibration"
  value="Absolute" arraysize="*"><DESCRIPTION>Type of spectral coord calibration</DESCRIPTION>
</PARAM>
<PARAM ID="SpectralResolution" datatype="double" name="SpectralResolution" ucd="spect.resolution" utype="spec:Spectrum.Char.SpectralAxis.Resolution" value="0">
<DESCRIPTION>Spectral resolution FWHM</DESCRIPTION>
</PARAM>
</GROUP>
<GROUP ID="Char.TimeAxis" name="Char.TimeAxis" utype="spec:Char.TimeAxis">
<DESCRIPTION>Time Axis Characterization</DESCRIPTION>
<PARAM ID="TimeAxisName" datatype="char" name="TimeAxisName" utype="spec:Spectrum.Char.TimeAxis.Name" value="Time" arraysize="*">
<DESCRIPTION>Name for time axis</DESCRIPTION>
</PARAM>
<PARAM ID="TimeAxisUcd" datatype="char" name="TimeAxisUcd" utype="spec:Spectrum.Char.TimeAxis.Ucd" value="time" arraysize="*">
<DESCRIPTION>UCD for time</DESCRIPTION>
</PARAM>
<PARAM ID="TimeAxisUnit" datatype="char" name="TimeAxisUnit" utype="spec:Spectrum.Char.TimeAxis.Unit" value="s" arraysize="*">
<DESCRIPTION>Unit for time</DESCRIPTION>
</PARAM>
<PARAM ID="TimeLocation" unit="d" datatype="double" name="TimeLocation" ucd="time.epoch" utype="spec:Spectrum.Char.TimeAxis.Coverage.Location.Value"
  value="51999.54882939815"><DESCRIPTION>Midpoint of exposure on MJD scale</DESCRIPTION>
</PARAM>
<PARAM ID="TimeExtent" datatype="double" name="TimeExtent" ucd="time.duration;obs.exposure" utype="spec:Spectrum.Char.TimeAxis.Coverage.Bounds.Extent" value="2701">
<DESCRIPTION>Total exposure time</DESCRIPTION>
</PARAM>
<PARAM ID="TimeStart" datatype="double" name="TimeStart" ucd="time.start;obs.exposure" utype="spec:Spectrum.Char.TimeAxis.Coverage.Bounds.Start"
 value="51999.52502210648"><DESCRIPTION>Start time</DESCRIPTION>
</PARAM>
<PARAM ID="TimeStop" datatype="double" name="TimeStop" ucd="time.end;obs.exposure" utype="spec:Spectrum.Char.TimeAxis.Coverage.Bounds.Stop" 
 value="51999.56039247685"><DESCRIPTION>Stop time</DESCRIPTION>
</PARAM>
</GROUP>
<GROUP ID="Char.FluxAxis" name="Char.FluxAxis" utype="spec:Char.FluxAxis">
<DESCRIPTION>Flux Axis Characterization</DESCRIPTION>
<PARAM ID="FluxAxisName" datatype="char" name="FluxAxisName" utype="spec:Spectrum.Char.FluxAxis.Name" value="Flux" arraysize="*">
<DESCRIPTION>Name for flux</DESCRIPTION>
</PARAM>
<PARAM ID="FluxAxisUcd" datatype="char" name="FluxAxisUcd" utype="spec:Spectrum.Char.FluxAxis.Ucd" value="phot.flux.density;em.wl" arraysize="*">
<DESCRIPTION>UCD for flux</DESCRIPTION>
</PARAM>
<PARAM ID="FluxAxisUnit" datatype="char" name="FluxAxisUnit" utype="spec:Spectrum.Char.FluxAxis.Unit" value="10**(-17) erg/cm**2/s/angstrom" arraysize="*">
<DESCRIPTION>Unit for flux</DESCRIPTION>
</PARAM>
<PARAM ID="FluxStatError" datatype="double" name="FluxStatError" ucd="stat.error;phot.flux" utype="spec:Spectrum.Char.FluxAxis.Accuracy.StatError" value="0">
<DESCRIPTION>Flux statistical error</DESCRIPTION>
</PARAM>
<PARAM ID="FluxCalibration" datatype="char" name="FluxCalibration" utype="spec:Spectrum.Char.FluxAxis.Accuracy.Calibration" value="Absolute" arraysize="*">
<DESCRIPTION>Type of flux calibration</DESCRIPTION>
</PARAM>
</GROUP>
\end{verbatim}
\end{fmlpage}

\begin{fmlpage}
\begin{verbatim}

<GROUP ID="Data.SpectralAxis" name="Data.SpectralAxis" utype="spec:Data.SpectralAxis">
<DESCRIPTION>Spectral Axis Data</DESCRIPTION>
<FIELDref ref="DataSpectralValue"/>
<FIELDref ref="DataSpectralBinLow"/>
<FIELDref ref="DataSpectralBinHigh"/>
<PARAM ID="DataSpectralUcd" datatype="char" name="DataSpectralUcd" utype="spec:Spectrum.Data.SpectralAxis.Ucd" value="em.wl" arraysize="*">
<DESCRIPTION>UCD for spectral coord</DESCRIPTION>
</PARAM>
<PARAM ID="DataSpectralUnit" datatype="char" name="DataSpectralUnit" utype="spec:Spectrum.Data.SpectralAxis.Unit" value="angstrom" arraysize="*">
<DESCRIPTION>Unit for spectral coord</DESCRIPTION>
</PARAM>
</GROUP>
<GROUP ID="Data.FluxAxis" name="Data.FluxAxis" utype="spec:Data.FluxAxis">
<DESCRIPTION>Flux Axis Data</DESCRIPTION>
<FIELDref ref="DataFluxValue"/>
<FIELDref ref="DataFluxStatErrLow"/>
<FIELDref ref="DataFluxStatErrHigh"/>
<FIELDref ref="DataFluxQuality"/>
<PARAM ID="DataFluxUcd" datatype="char" name="DataFluxUcd" utype="spec:Spectrum.Data.FluxAxis.Ucd" value="phot.flux.density;em.wl" arraysize="*">
<DESCRIPTION>UCD for flux</DESCRIPTION>
</PARAM>
<PARAM ID="DataFluxUnit" datatype="char" name="DataFluxUnit" utype="spec:Spectrum.Data.FluxAxis.Unit" value="10**(-17) erg/cm**2/s/angstrom" arraysize="*">
<DESCRIPTION>Unit for flux</DESCRIPTION>
</PARAM>
</GROUP>


<DATA>
<TABLEDATA>
<TR><TD>3822.0819974268243</TD><TD>3821.6419893046409</TD><TD>3822.5220562097306</TD><TD>0.6790500283241272</TD><TD>0</TD><TD>0</TD><TD>83886080</TD></TR>
<TR><TD>3822.9621656591976</TD><TD>3822.5220562097306</TD><TD>3823.40232578105</TD><TD>0.67889797687530518</TD><TD>0</TD><TD>0</TD><TD>83886080</TD></TR>
<TR><TD>3823.8425365811263</TD><TD>3823.40232578105</TD><TD>3824.2827980652619</TD><TD>0.67874801158905029</TD><TD>0</TD><TD>0</TD><TD>83886080</TD></TR>
<TR><TD>3824.7231102392952</TD><TD>3824.2827980652619</TD><TD>3825.1634731090558</TD><TD>0.67860102653503418</TD><TD>0</TD><TD>0</TD><TD>83886080</TD></TR>
<TR><TD>3825.6038866803838</TD><TD>3825.1634731090558</TD><TD>3826.0443509591169</TD><TD>0.67845600843429565</TD><TD>0</TD><TD>0</TD><TD>83886080</TD></TR>
<TR><TD>3826.4848659510972</TD><TD>3826.0443509591169</TD><TD>3826.9254316621559</TD><TD>0.67831301689147949</TD><TD>0</TD><TD>0</TD><TD>83886080</TD></TR>
<TR><TD>3827.3660480981362</TD><TD>3826.9254316621559</TD><TD>3827.8067152648787</TD><TD>0.6781730055809021</TD><TD>0</TD><TD>0</TD><TD>16777216</TD></TR>
<TR><TD>3828.2474331682283</TD><TD>3827.8067152648787</TD><TD>3828.6882018140186</TD><TD>0.67803502082824707</TD><TD>0</TD><TD>0</TD><TD>16777216</TD></TR>
...
<TR><TD>9200.25786648233</TD><TD>9199.1987086227764</TD><TD>9201.3171462889586</TD><TD>2.4964599609375</TD><TD>0</TD><TD>0</TD><TD>16777216</TD></TR>
<TR><TD>9202.37654805671</TD><TD>9201.3171462889586</TD><TD>9203.4360717996115</TD><TD>2.4953100681304932</TD><TD>0</TD><TD>0</TD><TD>16777216</TD></TR>
<TR><TD>9204.4957175317122</TD><TD>9203.4360717996115</TD><TD>9205.55548526707</TD><TD>2.4940199851989746</TD><TD>0</TD><TD>0</TD><TD>16777216</TD></TR>
<TR><TD>9206.61537501971</TD><TD>9205.55548526707</TD><TD>9207.6753868036922</TD><TD>2.4925899505615234</TD><TD>0</TD><TD>0</TD><TD>16777216</TD></TR>
<TR><TD>9208.7355206330667</TD><TD>9207.6753868036922</TD><TD>9209.7957765218853</TD><TD>2.4910099506378174</TD><TD>0</TD><TD>0</TD><TD>16777216</TD></TR>
</TABLEDATA>
</DATA>
</TABLE>
</RESOURCE>
</VOTABLE>
\end{verbatim}
\end{fmlpage}
\end{flushleft}
}
\end{landscape}

