{  \footnotesize
\begin{verbatim}
XTENSION= 'BINTABLE'           / binary table extension
BITPIX  =                    8 / 8-bit bytes
NAXIS   =                    2 / 2-dimensional binary table
NAXIS1  =                57344 / width of table in bytes
NAXIS2  =                    1 / number of rows in table
PCOUNT  =                    0 / size of special data area
GCOUNT  =                    1 / one data group (required keyword)
TFIELDS =                    7 / number of fields in each row
EXTNAME = 'SPECTRUM '           / name of this binary table extension
VOCLASS = 'Spectrum V1.0'      / VO Data Model
DATALEN =        180           / Segment size
VOSEGT  = 'Spectrum'           / Segment type
VOCSID  = 'MY-ICRS-TOPO'          / Coord sys ID
RADECSYS= 'FK5    '           / Not default - usually ICRS
EQUINOX =  2.0000000000000E+03 / default
TIMESYS = 'TT     '           / Time system
MJDREF  =  0.0                 / [d] MJD zero point for times
SPECSYS = 'TOPOCENT'            / Wavelengths are as observed
VOPUB   = 'CfA Archive'        / VO Publisher authority
VOREF   = '2006ApJ...999...99X'  / Bibcode for citation
VOPUBID = 'ivo://cfa.harvard.edu' / VO Publisher ID URI
VOVER   = '1.0'                   / VO Curation version
CONTACT = 'Jonathan McDowell, CfA'/
EMAIL   = 'jcm@cfa.harvard.edu'   /
VORIGHTS= 'public'  /
VODATE  = '2004-08-30'     /
DS_IDPUB= 'ivo://cfa.harvard.edu/spec#10304' / Publisher DID for dataset
COMMENT  DS_IDPUB usually the same as DS_IDENT?
OBJECT  = 'ARP 220 '           / Source name
OBJDESC = 'Merging galaxy Arp 220' / Source desc
SRCCLASS= 'Galaxy'             /
SPECTYPE= 'ULIRG'              /
REDSHIFT=              0.01812  / Emission redshift
RA_TARG      =     233.73791700     / [deg] Observer's specified target RA
DEC_TARG     =      23.50333300     / [deg] Observer's specified target Dec
TARGVAR =    0.2    /  20 percent variability amplitude
TITLE   = 'Observations of Merging Galaxies' /
AUTHOR  = 'MMT Archive'       / VO Creator
COLLECT1= 'Misc Pointed Observations'  / Collection
DS_IDENT= 'ivo://cfa.harvard.edu/spec#10304' / Publisher DID for dataset
CR_IDENT= 'ivo://cfa.harvard.edu/tdc#MMT4302-102' / Creator internal ID for dataset
DATE    = '2004-08-30T14:18:17' / Date and time of file creation
VERSION =  2                     / Reprocessed 2004 Aug
TELESCOP= 'MMT '              / Telescope  [Not part of Spectrum DM]
INSTRUME= 'MMT/BCS '          / Instrument
FILTER  = 'G220    '          / Grating  [Not part of Spectrum DM]
CRETYPE = 'Archival'          / Not an on-the-fly dataset
VOLOGO  = 'http://cfa.harvard.edu/vo/cfalogo.jpg' / VO Creator logo
CONTRIB1= 'Jonathan McDowell'  / Contributor
CONTRIB2= 'Wilhelm Herschel'  / Contributor
CONTRIB3= 'Harlow Shapley'     / Contributor
DSSOURCE= 'Pointed'   /  Survey or pointed, etc
DER_SNR =         5.0  /   Estimate of signal-to-noise
DER_Z   =       0.01845   /  Redshift measured in this spectrum
DER_ZERR =       0.00010   /  Error in DER_Z
TIMESDIM= 'T'                  / Time SIDim 
SPECSDIM= '10-10 L'             / Spectral SIDim
FLUXSDIM= '10+7 ML-1T-3'       / Flux SDim
SYS_ERR =        0.05            / Fractional systematic error in flux
FLUX_CAL= 'Calibrated'   /
SPEC_ERR=        0.01 / Stat error in spec coord, in SPEC units
SPEC_SYE=        0.001   /   Frac sys error in spec coord
SPEC_CAL= 'Calibrated'
SPEC_RES=              5.0  / [angstrom] Spectral resolution
SPECBAND= 'Optical'    / SED.Bandpass
SPEC_RP =             800.0  / Spectral resolving power
SPEC_VAL=            4100.0  / [angstrom]  Characteristic spec coord
SPEC_BW =            1800.0  / [angstrom]  Width of spectrum
SPEC_FIL=               1.0  / No gaps between channels
TIME_CAL = 'Calibrated'   /
DATE-OBS= '2004-06-03T21:18:17' / Date and time of observation
EXPOSURE =   1500.015         / [s] Effective exposure time
TSTART   =  52984.301203      / [d] MJD
TSTOP    =  52984.318564      / [d] MJD
TMID     =  52984.309883     / [d] MJD mid expsoure
SKY_CAL  = 'Calibrated'  /
SKY_RES =              1.0  / [arcsec] Spatial.Resolution
RA      =     233.73791       / [deg] Pointing position
DEC     =      23.50333       / [deg] Pointing position
APERTURE=       2.0           / [arcsec]  Aperture diameter/Slit width
TIME     =  52984.309883      / [d] MJD of midpoint


COMMENT  ---------------------------
COMMENT  WCS Paper 3 Keywords
1S4_1    = 'WAVE'            / Column name with spectral coord
1CTYP4   = 'WAVE-TAB'        / Spectral coord is WAVE 
1S5_1    = 'WAVE'            / Column name with spectral coord
1CTYP5   = 'WAVE-TAB'        / Spectral coord is WAVE 
1S6_1    = 'WAVE'            / Column name with spectral coord
1CTYP6   = 'WAVE-TAB'        / Spectral coord is WAVE 
1S7_1    = 'WAVE'            / Column name with spectral coord
1CTYP7   = 'WAVE-TAB'        / Spectral coord is WAVE 
COMMENT  ---------------------------
TTYPE1 = 'WAVE'   / Wavelength
TFORM1 = '180E'
TUNIT1 = 'angstrom'
TUCD1  = 'em.wl'                    /
TDMIN1 = 3195.0 /
TDMAX1 = 5005.0 /
TUTYP1 = 'Spectrum.Data.SpectralAxis.Value'
TTYPE2 = 'WAVE_LO' /
TFORM2 = '180E'
TUNIT2 = 'angstrom'
TUTYP2 = 'Spectrum.Data.SpectralAxis.Accuracy.StatErrLow'
TTYPE3 = 'WAVE_HI' /
TFORM3 = '180E'
TUNIT3 = 'angstrom'
TUTYP3 = 'Spectrum.Data.SpectralAxis.Accuracy.StatErrHigh'
TTYPE4 = 'FLUX' /
TFORM4 = '180E'
TUNIT4 = 'erg cm**(-2) s**(-1) angstrom**(-1)'
TUTYP4 = 'Spectrum.Data.FluxAxis.Value'
TUCD4  = 'phot.fluDens;em.wl'     / Type of Y axis: F-lambda
TTYPE5 = 'ERR_LO' /
TFORM5 = '180E'
TUNIT5 = 'erg cm**(-2) s**(-1) angstrom**(-1)'
TUTYP5 = 'Spectrum.Data.FluxAxis.Accuracy.StatErrLow'
TTYPE6 = 'ERR_HI' /
TFORM6 = '180E'
TUNIT6 = 'erg cm**(-2) s**(-1) angstrom**(-1)'
TUTYP6 = 'Spectrum.Data.FluxAxis.Accuracy.StarErrHigh'
TTYPE7 = 'QUALITY' /
TFORM7 = '180I'
TUTYP7 = 'Spectrum.Data.FluxAxis.Quality'
\end{verbatim}
}

The data would look like
{\small
\begin{verbatim}

WAVE   WAVE_LO WAVE_HI FLUX  ERR_LO  ERR_HI  QUALITY
3200.0 3195.0 3205.0 1.48E-12 2.0E-14 2.0E-14   0
3210.0 3205.0 3215.0 1.52E-12 3.0E-14 3.0E-14   0
3220.0 3215.0 3225.0 0.38E-12 0.38E-12 0.0      0                  
3230.0 3225.0 3235.0 1.62E-12 3.0E-14 3.0E-14   0
...
5000.0 4995.0 5005.0 1.33E-11 3.0E-13 3.0E-13   1
\end{verbatim}
}
