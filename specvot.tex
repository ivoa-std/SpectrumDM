
\section{VOTABLE serialization}

\subsection{Mapping Schema to VOTABLE}

We reproduce below the XML schema instance example as a VOTABLE instance example.
To go from the XML instance to the VOTABLE
instance, we:
\begin{itemize}
\item  - map the top level element to a RESOURCE
\item  - map all elements with simple content to PARAM
\item  - map all elements with complex content to GROUP
\item  - map the element names (with appropriate path) to values of the
utype attribute,
\item  - but, handle the FIELDS and Data elements in a special way.
The FIELDS element is used to define the table fields and the Data
element is used to define the table data. 
\item  - but, also,  all the second level elements below RESOURCE
except SPECTRUM map to an initial TABLE, while we map SPECTRUM 
to a second TABLE.
\item  - most of the elements extend the Param element, to which I have added
an optional name attribute that I have not used in the instance.
If this attribute is used, it can hold the name attributes of the PARAM and
FIELD; otherwise the relevant attributes could be filled with the
same value as the utype (without namespace prefix).

\end{itemize}

How can this be generalized to mapping an arbitrary data model
schema to VOTABLE? The only tricky parts are 
\begin{itemize}
\item  {\bf Spotting where the tabledata parts are. } 
We could require any DM schema that maps to VOTABLE
to include elements called FIELDS and Data (perhaps ROWS would be 
a better name), otherwise you would get a VOTABLE with no data section.
\item  {\bf Spotting where to start the main TABLE (i.e. the fact that
SPECTRUM is special). }  We could change the schema to
have an explicit attribute, annotation or other marker to tell us this.
\end{itemize}

These issues will require further discussion for future models.

\subsection{A VOTABLE instance}

The VOTable version of Spectrum uses a single VOTable {\lcaret}TABLE{\rcaret}
(Note that this may appear as one of many tables within an SED VOTable).
The data model fields described above as arrays map to
VOTable FIELDs, while the remaining fields map to PARAM.

We use nested GROUP constructs to delimit data model objects within the
main object, and PARAM and FIELD tags for attributes.
The nesting beyond a single GROUP is optional, as for cases for which
the utypes are unique within a group, the utypes can be used to infer
the datamodel structure. See
http://webtest.aoc.nrao.edu/ivoa-dal for a service returning VOTABLE
Spectrum instances with only one level of GROUP.

Names of fields and parameters are left to the data provider.
The utype and ucd attributes are used to denote data model and UCD tags.
The schema and namespace for the utypes is the XML schema given in section 8.4.
We have made up arbitrary NAME attributes for the PARAM and these
are not to be considered standard; the name fields are free
to be whatever the data provider wants, allowing compatibility with
local archive nomenclature. The NAME attributes for the FIELD elements
are also not standardized (of course they must be the same as in the
matching FIELDrefs); it is the utype attribute which is standardized.

The one departure from the XML schema below is that the `Data'
element and the individual `Point' elements are implicitly represented
by the table structure itself. Perhaps a UTYPE attribute to the
TABLEDATA element could be used to make this explicit.

The examples below describe a single SPECTRUM.



{ \footnotesize
\begin{flushleft}


\begin{fmpage}

\begin{verbatim}
<?xml version="1.0" encoding="UTF-8"?>
<VOTABLE version="1.1"
  xmlns:xsi="http://www.w3.org/2001/XMLSchema-instance"
  xsi:noNamespaceSchemaLocation="xmlns:http://www.ivoa.net/xml/VOTable/VOTable-1.1.xsd" 
  xmlns:spec="http://www.ivoa.net/xml/SpectrumModel/v1.01"
  xmlns="http://www.ivoa.net/xml/VOTable/v1.1">
<RESOURCE utype="spec:Spectrum">

<TABLE utype="spec:Spectrum">   
<GROUP utype="spec:Target">
 <PARAM name="Target" utype="spec:Target.Name" datatype="char" arraysize="*" value="Arp 220"/>
 <PARAM name="TargetPos" utype="spec:Target.pos" unit="deg" datatype="double" 
                                 arraysize="2" value="233.737917 23.503330"/>
 <PARAM name="z" utype="spec:Target.redshift" datatype="float" value="0.0018"/>
</GROUP>

<!-- SegmentType can be Photometry, TimeSeries or Spectrum -->
<PARAM name="Segtype" utype="spec:SegmentType" datatype="char" arraysize="*" 
                                                  value="Photometry" ucd="meta.code"/>
<GROUP name="CoordSys" utype="spec:CoordSys">
 <GROUP utype="CoordSys.SpaceFrame">
   <PARAM name="System" utype="spec:CoordSys.SpaceFrame.Name" ucd="pos.frame" 
                                         datatype="char" arraysize="*" value="ICRS"/>
   <PARAM name="Equinox" utype="spec:CoordSys.SpaceFrame.Equinox" ucd="time.equinox;pos.eq" 
                                         datatype="float" value="2000.0" />
 </GROUP>
 <GROUP utype="spec:CoordSys.TimeFrame">
  <PARAM name="TimeFrame" utype="spec:CoordSys.TimeFrame.Name" ucd="time.scale" datatype="char" 
                                       arraysize="*" value="UTC"/>
 </GROUP>
 <GROUP utype="spec:CoordSys.SpectralFrame">
  <PARAM name="SpectralFrame" utype="spec:CoordSys.SpectralFrame.RefPos" ucd="sdm:spect.frame" 
                                      datatype="char" arraysize="*" value="BARYCENTER"/>
 </GROUP>
</GROUP>

\end{verbatim}
\end{fmpage}

\begin{fmpage}
\begin{verbatim}


<GROUP utype="spec:Char">
 <GROUP utype="spec:Char.SpatialAxis">
   <PARAM name="SpatialAxisName" utype="name" ucd="pos.eq" unit="deg" value="Sky"/>
   <GROUP utype="spec:Char.SpatialAxis.Coverage">
    <GROUP utype="spec:Char.SpatialAxis.Coverage.Location">
     <PARAM name="SkyPos" utype="Char.SpatialAxis.Coverage.Location.Value" 
                                   ucd="pos.eq" unit="deg" 
                                   datatype="double" arraysize="2" value="132.4210 12.1232"/>
   </GROUP>
   <GROUP utype="Bounds">
     <PARAM name="SkyExtent" utype="Char.SpatialAxis.Coverage.Extent" ucd="pos.angDistance;instr.fov" 
                     datatype="double" unit="arcsec" value="20"/>
   </GROUP>
  </GROUP>
 </GROUP>



 <GROUP utype="spec:Char.TimeAxis">
  <PARAM name="TimeAxisName" utype="Char.TimeAxis.Name" ucd="time" unit="d" value="Time"/>
  <GROUP utype="Char.TimeAxis.Coverage">
   <GROUP utype="Char.TimeAxis.Coverage.Location">
    <PARAM name="TimeObs" utype="Char.TimeAxis.Coverage.Location.Value" ucd="time.epoch;obs"
                                   datatype="double" value="52148.3252"/>
   </GROUP>
   <GROUP utype="Char.TimeAxis.Coverage.Bounds">
    <PARAM name="TimeExtent" utype="Char.TimeAxis.Coverage.Bounds.Extent" ucd="time.duration"
                     unit="s" datatype="double" value="1500.0" />
    <PARAM name="TimeStart" utype="Char.TimeAxis.Coverage.Bounds.Start" ucd="time.start" unit="s" 
                                     datatype="double" value="52100.000" />
    <PARAM name="TimeStop" utype="Char.TimeAxis.Coverage.Bounds.Stop" ucd="time.end" unit="s" 
                                     datatype="double" value="52300.000" />
   </GROUP>
   <GROUP utype="Char.TimeAxis.Coverage.Support">
    <PARAM name="TimeExtent" utype="Char.TimeAxis.Coverage.Support.Extent" ucd="time.duration;obs.exposure"
                     unit="s" datatype="double" value="1500.0" />
    <PARAM name="TimeStart" utype="Char.TimeAxis.Coverage.Bounds.Start" ucd="time.start" unit="s" 
                                     datatype="double" value="52100.000" />
    <PARAM name="TimeStop" utype="Char.TimeAxis.Coverage.Bounds.Stop" ucd="time.end" unit="s" 
                                     datatype="double" value="52300.000" />
   </GROUP>

  </GROUP>
 </GROUP>

 <GROUP utype="spec:Char.SpectralAxis">
  <PARAM name="SpectralAxisName" utype="Char.SpectralAxis.Name" ucd="em.wl" unit="angstrom" value="Wavelength"/>
  <GROUP utype="Char.SpectralAxis.Coverage">
   <GROUP utype="Char.SpectralAxis.Coverage.Bounds">
    <PARAM name="SpectralExtent" utype="Char.SpectralAxis.Coverage.Bounds.Extent" ucd="instr.bandwidth" 
                     unit="angstrom" datatype="double" value="3000.0"/>
   </GROUP>
  </GROUP>
 </GROUP>
</GROUP>
\end{verbatim}
\end{fmpage}

\begin{fmpage}
\begin{verbatim}

<GROUP utype="spec:Curation">
 <PARAM name="Publisher" utype="spec:Curation.Publisher" ucd="meta.curation" 
                       datatype="char" arraysize="*" value="SAO"/>
 <PARAM name="PubID" utype="spec:Curation.PublisherID" ucd="meta.ref.url;meta.curation" datatype="char" 
                                       arraysize="*" value="ivo://cfa.harvard.edu"/>
 <PARAM name="Contact" utype="spec:Curation.Contact.Name" ucd="meta.bib.author;meta.curation" 
                       datatype="char" arraysize="*" value="Jonathan McDowell"/>
 <PARAM name="email" utype="spec:Curation.Contact.Email" ucd="meta.email" datatype="char" 
                                      arraysize="*" value="jcm@cfa.harvard.edu"/>
</GROUP>


<GROUP utype="spec:DataID">
 <PARAM name="Title" utype="spec:DataID.Title" datatype="char" arraysize="*" value="Arp 220 SED"/>
 <PARAM name="Creator" utype="spec:Segment.DataID.Creator" ucd="meta.curation" datatype="char" 
                                      arraysize="*" value="ivo://sao/FLWO"/>
 <PARAM name="DataDate" utype="spec:DataID.Date" ucd="time.epoch;meta.dataset"
                      datatype="char" arraysize="*" value="2003-12-31T14:00:02Z"/>
 <PARAM name="Version" utype="spec:DataID.Version" ucd="meta.version;meta.dataset"
                      datatype="char" arraysize="*" value="1"/>
 <PARAM name="Instrument" utype="spec:DataID.Instrument" ucd="meta.id;instr" datatype="char" 
                                    arraysize="*" value="BCS"/>
 <PARAM name="Filter" utype="spec:DataID.Collection" ucd="inst.filter.id" datatype="char" 
                                   arraysize="*" value="G300"/>
 <PARAM name="CreationType" utype="spec:DataID.CreationType" datatype="char" arraysize="*" value="Archival"/>
 <PARAM name="Logo" utype="spec:DataID.Logo" ucd="meta.ref.url" datatype="char" 
                                        arraysize="*" value="http://cfa-www.harvard.edu/nvo/cfalogo.jpg"/>
</GROUP>

<GROUP utype="spec:Derived">
 <PARAM name="SNR" utype="spec:Derived.SNR" datatype="float" value="3.0"/>
</GROUP>

<GROUP utype="spec:Data">

<GROUP utype="spec:Data.SpectralAxis">
 <FIELDref ref="Coord"/>

 <GROUP utype="spec:Data.SpectralAxis.Accuracy">
  <FIELDref ref="BinLow"/>
  <FIELDref ref="BinHigh"/>
 </GROUP>
<!-- In this case Resolution is demoted from Field to Param since it is constant -->
 <PARAM name="Resolution" utype="spec:Data.SpectralAxis.Resolution" 
      ucd="spect.resolution;em.wl"    unit="angstrom" datatype="float" value="14.2"/>
</GROUP>

\end{verbatim}
\end{fmpage}

\begin{fmpage}
\begin{verbatim}
<GROUP utype="spec:Data.FluxAxis">
 <FIELDref ref="Flux1"/>
 <GROUP utype="spec:Data.FluxAxis.Accuracy">
  <FIELDref ref="ErrorLow"/>
  <FIELDref ref="ErrorHigh"/>
  <PARAM name="SysErr" utype="SysErr" unit="" datatype="float" value="0.05"/>
 </GROUP>
 <FIELDref ref="Quality"/>
</GROUP>
</GROUP>

<FIELD name="Coord" ID="Coord" utype="spec:Data.SpectralAxis.Value" ucd="em.wl"
              datatype="double" unit="angstrom"/>
<FIELD name="BinLow" ID="BinLow" utype="spec:Data.SpectralAxis.BinLow" 
              ucd="em.wl;stat.min"
              datatype="double" unit="angstrom"/>
<FIELD name="BinHigh" ID="BinHigh" utype="spec:Data.SpectralAxis.BinHigh"
               ucd="em.wl;stat.max"
              datatype="double" unit="angstrom"/>
<FIELD name="Flux" ID="Flux1" utype="spec:Data.FluxAxis.value" ucd="phot.flux.density;em.wl"
              datatype="double" unit="erg cm**(-2) s**(-1) angstrom**(-1)"/>
<FIELD name="ErrorLow" ID="ErrorLow" utype="spec:Data.FluxAxis.Accuracy.StatErrLow" 
              datatype="double" unit="erg cm**(-2) s**(-1) angstrom**(-1)"/>
<FIELD name="ErrorHigh" ID="ErrorHigh" utype="spec:Data.FluxAxis.Accuracy.StatErrHigh" 
              datatype="double" unit="erg cm**(-2) s**(-1) angstrom**(-1)"/>
<FIELD name="Quality" ID="Quality" datatype="int" utype="spec:Data.FluxAxis.Quality"/>
<DATA>
<TABLEDATA>
<!-- Note slightly nonlinear wavelength solution -->
<!-- Second row is upper limit -->
<!-- Third row has quality mask set -->
<TR><TD>3200.0</TD><TD>3195.0</TD><TD>3205.0</TD><TD>1.38E-12</TD><TD>5.2E-14</TD><TD>6.2E-14</TD>
                                                                                    <TD>0</TD></TR>
<TR><TD>3210.5</TD><TD>3205.0</TD><TD>3216.0</TD><TD>1.12E-12</TD><TD>1.12E-12</TD>
                                                                     <TD>0</TD><TD>0</TD></TR>
<TR><TD>3222.0</TD><TD>3216.0</TD><TD>3228.0</TD><TD>1.42E-12</TD><TD>1.3E-14</TD>
                                                                     <TD>0.2E-14</TD><TD>3</TD></TR>
</TABLEDATA>
</DATA>
</TABLE>
</RESOURCE>
</VOTABLE>

\end{verbatim}
\end{fmpage}

\end{flushleft}
}

A second example, based on the reference SSAP proxy service for
the JHU SDSS spectrum archive:

\input{doug.vot}
