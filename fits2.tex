

\section{FITS serialization}


\subsection{Mapping Spectrum to FITS}


\paragraph {\bf FITS serialization design:}
We define a reference serialization of this data model as a FITS binary table.
The table represents a spectrum or photometry point as a single row of a table.

This serialization is a special case of an SED (or spectral association)
serialization which
uses one row per spectral segment; in that case,
variable-length arrays may be used to contain the array quantities.
In each case below where a `variable length array' is specified, fixed length
arrays are suitable for a single spectrum or for multiple spectra where
all the arrays are the same length, but readers should be
prepared to handle the variable length case. 


For SEDs, another approach would be to have one FITS HDU per spectrum or
photometry point. However this was rejected as unworkable, as the
overhead of 5760 bytes (2 FITS blocks) per photometry point would
inflate the data for the photometry-only SED case by factors of around
50-100.

\paragraph {\bf Standard table keywords:}
In table F.1 we give the mapping of data model fields to FITS columns and keywords.
For each column, the standard keywords TTYPEn, TUNITn, TFORMn should
be provided. Order of keywords and columns is not significant,
except that it is strongly recommended that RA and Dec be in adjacent columns
or keywords. Additional keywords and columns which are not part of the
model (including other conventions such as e.g. TDMINn)  are allowed 
to be present, but are not guaranteed to be propagated by VO software.

\paragraph {\bf Keywords and columns: `Greenbank' convention}

In Table F.1 we give single metadata items as keywords; arrays
of data (members of the Spectrum.Data classes) are stored as
columns, and Table F.1 gives the column name, i.e. the value
of the keyword TTYPEn. The 'Source' column in Table F.1 indicates
if the name (if keyword) or value (if column) 
is a FITS standard (S), an existing convention (C) such
as one of the HEA conventions, or is newly invented (N).

In some cases, the column data arrays may have the same value for each
data point. In this case we may use the 'Greenbank' convention in which
the column is omitted and replaced by a keyword whose name is the same
as the column.  Further, in SED applications when multiple spectrum
data lines are present, some metadata may differ from line to line and
be promoted from keyword to column.
Therefore, implementors should  check both keywords and
column names for the appropriate tokens.

%We recommend that the same token NOT be used for a keyword and 
%a column name in the same FITS HDU.




\paragraph {\bf TUTYP and TUCD keywords:}
We map the FITS columns to the model by using TUTYPn keywords.
TUTYPn (string-valued) gives the data model field name (UTYPE string)
for the data in column n. Thus, the x and y axes (i.e. spectral coordinate
and flux-like axes) of the spectrum
have TUTYPn value of Spectrum.Data.SpectralAxis.Value and
Spectrum.Data.FluxAxis.Value respectively. 


Different kinds of x and y axis are identified by the
Spectrum.Data.SpectralAxis.UCD and Spectrum.Data.FluxAxis.UCD data model
fields, which are mapped to TUCDn keywords. TUCDn (string valued) gives
the UCD corresponding to the data in column n. Both TUTYPn and TUCDn
should be present for any column which corresponds to a Spectrum data model
field; they are optional for any additional data columns which are not part
of the Spectrum model.
The units of spectral coordinate and flux
are given in the TUNITn keys of the corresponding data columns.
There is no separate provision for units of Char.SpectralAxis or
Char.FluxAxis; these are required to be the same as for the data.


The TTYPEn keywords
for the x and y columns are free, but it is strongly recommended that
(for consistency of style with WCS Paper 3) the values for the
x axis have for their first 4 characters 'WAVE', 'FREQ' and 'ENER'
for the case of wavelength, frequency and energy respectively.
We also recommend the value 'FLUX' for the y axis, where appropriate.
Nevertheless, it is the TUTYPn and TUCDn keywords that should be
used to interpret the semantics of the file.


\paragraph {\bf WAVE, ENER, and FREQ}

In the header metadata, such as the Spectrum.Char entries,
we use SPEC\_ keywords to denote the spectral axis generically, but
in the table columns (Spectrum.Data entries) we use the
terms WAVE\_, ENER\_, and FREQ\_ as appropriate. 
Thus if the Spectrum.Data.SpectralAxis.Value field is WAVE,
the SpectralAxis.Accuracy.BinLow field should be WAVE\_LO;
if Value is FREQ, BinLow should be FREQ\_LO. 
We believe the small extra parsing overhead is worth it
for the readability and interoperability (since these names have
been used in existing FITS files) of the crucial main data table.


\paragraph {\bf Char and Data keywords:}
The model contains both Characterization metadata, giving
overall typical values for quantities such as spectral resolution,
and the Data object, which can include such quantities on
a per-pixel basis. In some cases, the FITS serialization allows
the same token for both Char (as a keyword) and Data (as a column name).
The name, unit and UCD fields for Char.FluxAxis and Char.SpectralAxis
are required to be the same as for Data.FluxAxis and Data.SpectralAxis.
The case of TimeAxis is a little different, since there may be no
Data.TimeAxis present, and there exist already some HEA conventions for
recording TimeAxis characterization, notably the TIMEUNIT keyword.
Note that TIMESYS, if present, must be TT.

\paragraph {\bf VOCLASS keyword:}
We add a new keyword VOCLASS to describe the VO object represented by the
FITS table. The value of VOCLASS should be 'SPECTRUM 1.00'.


\paragraph {\bf WCS table keywords:}
The spectral coordinate
may also be identified by optional 1Sn\_1 and 1CTYPn keywords
as per WCS Paper 3. Table 9 of that paper implies that each data column
which is a function of the spectral coord needs a pair of such keywords.
Applications which implement the spectrum data model may ignore the WCS keys and interpret
the file by recognizing 'by spec' (using TUTYPn) 
which column is the spectral coordinate and that
FLUX, etc. are functions of it, but the WCS keys give a general FITS application
a chance at making sense of the file. In the example, TTYPE5='ERR\_LO'
and TUTYP5='Spectrum.Data.FluxAxis.Accuracy.StatErrLow'; the WCS keyword
1CTYPE5='WAVE-TAB' indicates that the data in column 5 is a function
of wavelength, and that the wavelengths are in a lookup table. The WCS
keyword 1S5\_1='WAVE' indicates that the lookup table for the x-axis of
column 5 (in this case, the wavelengths that the ERR\_LO values correspond
to) is in the column with TTYPEn='WAVE', in this case column 1.

Note that APERTURE has also been used elsewere as string-valued
to indicate a named aperture; this is not allowed here.

The mid-exposure value is a required field for the internal data model;
however it can be calculated from TSTART and TSTOP if they are
present, and is then optional for the FITS serialization.
The dataset start and stop wavelength may be provided in
standard FITS as TDMINn/TDMAXn
where n is the number of the column with the wavelengths.

For FITS, CoordSys.SpaceFrame.ucd is required to be the
same as Char.SpatialAxis.ucd and CoordSys.TimeFrame.ucd
is required to be equal to the default value "time".
CoordSys.SpectralFrame.ucd is required to be the same
as Data.SpectralAxis.ucd, present as a TUCDn value.

To express the CoordSys.RedshiftFrame, we recommend using a FITS WCS
system with suffix 'Z' applied to the spectral coordinate axis, when
appropriate. 
CoordSys.RedshiftFrame.DopplerDefinition is represented by the first four characters
of TCTYPnZ and should have the values VRAD, VOPT, ZOPT or VELO,
as per the convention for spectral CTYPE keywords in Paper III of the FITS WCS system.
CoordSys.RedshiftFrame.RefPos is represented by SPECSYSZ and should have values
as listed in Paper III of the FITS WCS system.

%The Position.Type field is not explicitly serialized:
%the names of the two columns used for Position are used
%to infer the type. 

%In some cases, the characterization and data keywords are the same,
%e.g. SKY\_RES. The implication here is that if SKY\_RES is provided
%as a keyword, it represents the characterization and a typical value
%for the data; if it is provided as a column TTYPEn=SKY\_RES it
%represents individual values for the data, and is not provided
%as a generic characterization; the same token should not be a keyword
%name and a column name in the same dataset.

\vskip 0.2in

{\small

\colorbox{iblue}{
\begin{minipage}[l]{6.5in}
\begin{tabular}{llll} 
\hline
\multicolumn{3}{c}{Table F.1: FITS keywords for VO Spectrum}\\
\hline
Data model field & FITS keyword & Source & Value if fixed \\
\hline
 & & \\
DataModel   & VOCLASS  & N       & SPECTRUM 1.0\\
Length  & DATALEN & N\\
Type    & VOSEGT & N\\
CoordSys.ID  &VOCSID   & N\\
CoordSys.SpaceFrame.Name &RADECSYS   &S    & e.g. ICRS or FK5 \\
CoordSys.SpaceFrame.Equinox&EQUINOX  &S   & e.g. 2000.0 \\
CoordSys.SpaceFrame.ucd &SKY\_UCD &   & pos.eq \\
CoordSys.SpaceFrame.RefPos &SKY\_REF   &S    &\\
CoordSys.TimeFrame.Name &TIMESYS     &C   & TT  \\
CoordSys.TimeFrame.ucd& -    &C   & time \\
CoordSys.TimeFrame.Zero& MJDREF & C&default 0.0\\
CoordSys.TimeFrame.RefPos &   & & (not used) \\
CoordSys.SpectralFrame.RefPos &  SPECSYS&S & (see below)\\
CoordSys.SpectralFrame.ucd&  TUCDn &C   & = Data.SpectralAxis.ucd \\
CoordSys.SpectralFrame.Redshift & REST\_Z & N & \\
CoordSys.SpectralFrame.Name &SPECNAME  &  & \\
CoordSys.RedshiftFrame.Name &ZNAME  &  & \\
CoordSys.RedshiftFrame.DopplerDefinition &TCTYPnZ  &  & \\
CoordSys.RedshiftFrame.RefPos & SPECSYSZ & S  &   \\
Curation.Publisher &VOPUB &N \\
Curation.Reference& VOREF &N \\
Curation.PublisherID &VOPUBID&N    \\
Curation.Version&VOVER&N\\
Curation.ContactName&CONTACT&N \\
Curation.ContactEmail&EMAIL &N \\
Curation.Rights & VORIGHTS &N\\
Curation.Date  & VODATE  &N\\
Curation.PublisherDID  &DS\_IDPUB & N\\
Target.Name   & OBJECT       &S \\
Target.Description & OBJDESC&N\\
Target.Class  & SRCCLASS    &N \\
Target.spectralClass & SPECTYPE&N\\
Target.redshift& REDSHIFT  & C\\ 
Target.pos & RA\_TARG, DEC\_TARG& C\\
Target.VarAmpl & TARGVAR & N\\
DataID.Title    & TITLE    & C\\
DataID.Creator    & AUTHOR &S \\
DataID.Collection & COLLECTn& N\\ 
DataID.DatasetID& DS\_IDENT & N\\
DataID.CreatorDID& CR\_IDENT & N\\
DataID.Date    & DATE  & S\\
DataID.Version & VERSION& C \\
DataID.Instrument & INSTRUME & S\\
DataID.CreationType    & CRETYPE&N\\
DataID.Logo&VOLOGO       & N\\
DataID.Contributor&CONTRIBn&N\\
DataID.DataSource&DSSOURCE & N\\
DataID.Bandpass  & SPECBAND & N\\
\end{tabular}
\end{minipage}
}

\colorbox{iblue}{
\begin{minipage}[l]{6.5in}
\begin{tabular}{llll} 
\hline
Data model field & FITS keyword & Source & Value if fixed \\
\hline
 & & \\ 
Derived.SNR  & DER\_SNR  & N\\
Derived.redshift.value & DER\_Z & N\\
Derived.redshift.statError& DER\_ZERR&N\\
Derived.redshift.Confidence& DER\_ZCNF&N\\
Derived.VarAmpl  & DER\_VAR & N\\
TimeSI &TIMESDIM          &N     &\\
SpectralSI & SPECSDIM    & N&\\
FluxSI & FLUXSDIM        & N&\\
\\
\multicolumn{4}{c}{Omitted Char fields, values inherited from Spectrum.Data}\\
\\
Char.FluxAxis.Name & -    & -   & TTYPEn for FLUX \\
Char.FluxAxis.Unit & -    &-    & Same as Data\\
Char.FluxAxis.ucd &   - & -      & Same as Data\\
Char.SpectralAxis.Name & - & -      & Same as Data\\
Char.SpectralAxis.Unit & - & -      & Same as Data\\
Char.SpectralAxis.ucd   & - & -      & Same as Data\\
Char.TimeAxis.Name &  -  & -  & TIME \\
Char.TimeAxis.ucd  & -    & -& time\\
Char.SpatialAxis.Name   &  -  &  & (not used) \\
Char.SpatialAxis.Unit   &  -  &  & deg \\
\\
\multicolumn{4}{c}{Char Fields which are the same as for Spectrum.Data}\\
\\
Char.FluxAxis.Accuracy.StatError & STAT\_ERR &C\\
Char.FluxAxis.Accuracy.SysError & SYS\_ERR & C\\
%Spectrum.Char.FluxAxis.Quality & QUALITY & C \\
Char.TimeAxis.Accuracy.StatError & TIME\_ERR & N\\
Char.TimeAxis.Accuracy.SysError &  TIME\_SYE & N\\
Char.TimeAxis.Resolution & TIME\_RES  & N\\
\\
\end{tabular}
\end{minipage}
}


\colorbox{iblue}{
\begin{minipage}[l]{7.0in}
\begin{tabular}{lllp{1.5in}} 
Data model field & FITS keyword& Source & Value if fixed \\
\hline
 & & \\
\multicolumn{4}{c}{Char Fields which are only present in Char}\\
\\
Char.FluxAxis.Calibration & FLUX\_CAL& N\\
Char.SpectralAxis.Calibration & SPEC\_CAL&N \\
Char.SpectralAxis.Coverage.Location.Value & SPEC\_VAL &N\\
Char.SpectralAxis.Coverage.Bounds.Extent & SPEC\_BW & N\\ 
Char.SpectralAxis.Coverage.Bounds.Start  & TDMINn & \\
Char.SpectralAxis.Coverage.Bounds.Stop   & TDMAXn & \\
Char.SpectralAxis.SamplingPrecision. &&\\
 SamplingPrecisionRefVal.FillFactor & SPEC\_FIL & N\\
Char.SpectralAxis.SamplingPrecision. &&&\\
\quad SampleExtent& SPEC\_BIN & N & \\
Char.SpectralAxis.Accuracy.BinSize & SPEC\_BIN& N& \\
Char.SpectralAxis.Accuracy.StatError & SPEC\_ERR&N\\
Char.SpectralAxis.Accuracy.SysError & SPEC\_SYE& N\\
Char.SpectralAxis.Resolution &  SPEC\_RES & N\\
Char.SpectralAxis.ResPower &  SPEC\_RP & N\\
{Char.SpectralAxis.Coverage.Support.Extent  } & SPECWID & N\\
Char.TimeAxis.Unit   & TIMEUNIT & C\\
Char.TimeAxis.Accuracy.BinSize & TIMEDEL & C \\
Char.TimeAxis.Calibration & TIME\_CAL & N\\
Char.TimeAxis.Coverage.Location.Value & TMID & N\\
Char.TimeAxis.Coverage.Bounds.Extent& TELAPSE & C\\ 
Char.TimeAxis.Coverage.Bounds.Start & TSTART & C\\
Char.TimeAxis.Coverage.Bounds.Stop & TSTOP & C\\
Char.TimeAxis.Coverage.Support.Extent& EXPOSURE & C\\ 
Char.TimeAxis.SamplingPrecision. &&\\
 SamplingPrecisionRefVal.FillFactor & DTCOR & C\\
Char.TimeAxis.SamplingPrecision.     &&&\\
\quad SampleExtent &TIMEDEL & S & \\
Char.SpatialAxis.ucd   & SKY\_UCD& N & pos.eq \\
Char.SpatialAxis.Accuracy.StatErr & SKY\_ERR & N\\
Char.SpatialAxis.Accuracy.SysError &SKY\_SYE & N\\
Char.SpatialAxis.Calibration & SKY\_CAL&N\\
Char.SpatialAxis.Resolution & SKY\_RES & N\\
Char.SpatialAxis.Coverage.Location.Value & RA, DEC, etc.& C\\
Char.SpatialAxis.Coverage.Bounds.Extent & APERTURE & C\\
Char.SpatialAxis.Coverage.Support.Area& REGION & N & String value\\
Char.SpatialAxis.Coverage.Support.Extent & AREA &N & \\
Char.SpatialAxis.SamplingPrecision. &&\\
 SamplingPrecisionRefVal.FillFactor &SKY\_FILL &N\\
Char.SpatialAxis.SamplingPrecision.  &&&\\
\quad SampleExtent& TCDLTn &S &\\
\end{tabular}
\end{minipage}
}


\colorbox{iblue}{
\begin{minipage}[l]{7.0in}
\begin{tabular}{lllp{1.5in}} 
Data model field & FITS keyword& Source & Value if fixed \\
\hline
 & & \\
\hline
\multicolumn{4}{c}{ Per-data-point values }\\
\hline
\\
Data.FluxAxis.Value & TTYPEn & S & FLUX\\
UTYPE of above ...               & TUTYPn & N & 'Spectrum.Data.FluxAxis.Value'\\\
Data.FluxAxis.Unit & TUNITn& S\\
Data.FluxAxis.ucd &   TUCDn  & N& (same as Char)\\
Data.FluxAxis.Accuracy.StatError & TTYPEn  & N & ERR\\
Data.FluxAxis.Accuracy.StatErrLow & TTYPEn & C & ERR\_LO\\
Data.FluxAxis.Accuracy.StatErrHigh& TTYPEn& C & ERR\_HI\\ 
Data.FluxAxis.Accuracy.SysError & TTYPEn & C & SYS\_ERR \\
Data.FluxAxis.Quality & TTYPEn & C & QUALITY\\
Data.FluxAxis.QualityN & TTYPEn & C & QUALn\\
Data.SpectralAxis.Value  & TTYPEn & S & WAVE,ENER,FREQ\\
UTYPE of above ...               & TUTYPn & N & 'Spectrum.Data.SpectralAxis.Value'\\
Data.SpectralAxis.Unit & TUNITn & S & (same as Char)\\
Data.SpectralAxis.ucd &  TUCDn &N & (same as Char) \\
Data.SpectralAxis.Accuracy.BinSize& TTYPEn & N & WAVE\_BIN,ENER\_BIN, FREQ\_BIN\\
Data.SpectralAxis.Accuracy.BinLow & TTYPEn & N & WAVE\_LO,ENER\_LO, FREQ\_LO \\
Data.SpectralAxis.Accuracy.BinHigh& TTYPEn & N & WAVE\_HI,ENER\_HI, FREQ\_HI \\
Data.SpectralAxis.Accuracy.StatError & TTYPEn& N & WAVE\_ERR,ENER\_ERR, FREQ\_ERR \\
Data.SpectralAxis.Accuracy.StatErrLow& TTYPEn& N & WAVE\_ELO,ENER\_ELO, FREQ\_ELO \\
Data.SpectralAxis.Accuracy.StatErrHigh & TTYPEn& N& WAVE\_EHI,ENER\_EHI, FREQ\_EHI \\
Data.SpectralAxis.Accuracy.SysError & TTYPEn & N &  WAVE\_SYE,ENER\_SYE, FREQ\_SYE \\
Data.SpectralAxis.Resolution & TTYPEn & N & WAVE\_RES,ENER\_RES, FREQ\_RES \\
Data.TimeAxis.Value & TTYPEn & C & TIME\\
UTYPE of above ...               & TUTYPn & N & 'Spectrum.Data.TimeAxis.Value' \\
Data.TimeAxis.Unit &  TUNITn& S  & (same as Char)\\
Data.TimeAxis.ucd  &  TUCDn & N   & time \\
Data.TimeAxis.Accuracy.BinLow & TTYPEn & N & TIME\_LO \\
Data.TimeAxis.Accuracy.BinHigh & TTYPEn & N & TIME\_HI \\
Data.TimeAxis.Accuracy.BinSize & TIMEDEL  & S & \\
Data.TimeAxis.Resolution& TTYPEn & N  & TIME\_RES \\
Data.TimeAxis.Accuracy.StatError & TTYPEn & N & TIME\_ERR \\
Data.TimeAxis.Accuracy.StatErrLow & TTYPEn & N & TIME\_ELO \\
Data.TimeAxis.Accuracy.StatErrHigh& TTYPEn & N & TIME\_EHI \\
Data.TimeAxis.Accuracy.SysError & TTYPEn & N  & TIME\_SYE \\
Data.BackgroundModel.Value & TTYPEn & N  & BGFLUX \\
UTYPE of above ...               & TUTYPn & N & 'Spectrum.Data.BackgroundModel.Value'\\
Data.BackgroundModel.Unit & TUNITn  &S&  (same as FluxAxis) \\
Data.BackgroundModel.ucd & TUCDn & N &  (same as FluxAxis)   \\
Data.BackgroundModel.Accuracy.StatError & TTYPEn & N & BG\_ERR \\
Data.BackgroundModel.Accuracy.StatErrLow & TTYPEn & N & BG\_ELO \\
Data.BackgroundModel.Accuracy.StatErrHigh& TTYPEn & N & BG\_EHI \\ 
Data.BackgroundModel.Accuracy.SysError & TTYPEn & N & BG\_SYE \\
Data.BackgroundModel.Quality & TTYPEn & N & BGQUAL\\
\\
\end{tabular}
\end{minipage}
}
}
\vskip 0.2in
\clearpage

\subsection{Expressing the spectrum spatial coordinates in FITS}

FITS has a sophisticated mechanism for expressing celestial
coordinates. However, it applies only to image axes or table
columns. If you want to express a single celestial position in the
header of a FITS binary table, the WCS conventions do not apply.
You could add an extra pair of columns to the table giving the
same position in each row, but that would be wasteful.

Here we propose a local convention leveraging the existing
WCS conventions:
\begin{itemize}
\item The keyword names for the coordinates are those used
in the first four characters of the CTYPE values for the WCS paper:
e.g. RA, DEC, GLON, GLAT.
\item The coordinate system is identified by the keyword SKY\_UCD with
values such as pos.eq, etc.
\item The RADECSYS and EQUINOX keywords should be used when appropriate.
\item Values are always in degrees.
\item The VOCSID optional keyword is provided to allow VO coordinate
system names used in Spectrum and STC to be propagated to FITS.
Its value is not relevant in the FITS context.
\end{itemize}



\subsection{The SPECSYS keyword}

We note the allowed values of the SPECSYS keyword from
Greisen et al and the corresponding values from the VO STC:

\vskip 0.1in

\colorbox{iblue}{
\begin{tabular}{lll}
\hline
FITS   & STC   & Meaning\\
\hline
TOPOCENT &TOPOCENTER & Topocenter\\
GEOCENTR &GEOCENTER & Geocenter\\
BARYCENT &BARYCENTER & Solar System Barycenter\\
HELIOCEN &HELIOCENTER & Heliocenter\\
LSRK     &LSRK & Kinematic local standard of rest\\
LSRD     &LSRD & Dynamic local standard of rest\\
GALACTOC &GALACTIC\_CENTER & Galactic center\\
LOCALGRP &LOCAL\_GROUP\_CENTER & Local group barycenter\\
CMBDIPOL &- & Frame of the Cosmic Microwave Background dipole\\
SOURCE   &- & Source rest frame \\
\hline
\end{tabular}
}

\clearpage

\subsection{An instance example}


We summarize this with a sample FITS extension header.

{  \footnotesize
\begin{verbatim}
XTENSION= 'BINTABLE'           / binary table extension
BITPIX  =                    8 / 8-bit bytes
NAXIS   =                    2 / 2-dimensional binary table
NAXIS1  =                57344 / width of table in bytes
NAXIS2  =                    1 / number of rows in table
PCOUNT  =                    0 / size of special data area
GCOUNT  =                    1 / one data group (required keyword)
TFIELDS =                    7 / number of fields in each row
EXTNAME = 'SPECTRUM '           / name of this binary table extension
VOCLASS = 'Spectrum V1.0'      / VO Data Model
DATALEN =        180           / Segment size
VOSEGT  = 'Spectrum'           / Segment type
VOCSID  = 'MY-ICRS-TOPO'          / Coord sys ID
RADECSYS= 'FK5    '           / Not default - usually ICRS
EQUINOX =  2.0000000000000E+03 / default
TIMESYS = 'TT     '           / Time system
MJDREF  =  0.0                 / [d] MJD zero point for times
SPECSYS = 'TOPOCENT'            / Wavelengths are as observed
VOPUB   = 'CfA Archive'        / VO Publisher authority
VOREF   = '2006ApJ...999...99X'  / Bibcode for citation
VOPUBID = 'ivo://cfa.harvard.edu' / VO Publisher ID URI
VOVER   = '1.0'                   / VO Curation version
CONTACT = 'Jonathan McDowell, CfA'/
EMAIL   = 'jcm@cfa.harvard.edu'   /
VORIGHTS= 'public'  /
VODATE  = '2004-08-30'     /
DS_IDPUB= 'ivo://cfa.harvard.edu/spec#10304' / Publisher DID for dataset
COMMENT  DS_IDPUB usually the same as DS_IDENT?
OBJECT  = 'ARP 220 '           / Source name
OBJDESC = 'Merging galaxy Arp 220' / Source desc
SRCCLASS= 'Galaxy'             /
SPECTYPE= 'ULIRG'              /
REDSHIFT=              0.01812  / Emission redshift
RA_TARG      =     233.73791700     / [deg] Observer's specified target RA
DEC_TARG     =      23.50333300     / [deg] Observer's specified target Dec
TARGVAR =    0.2    /  20 percent variability amplitude
TITLE   = 'Observations of Merging Galaxies' /
AUTHOR  = 'MMT Archive'       / VO Creator
COLLECT1= 'Misc Pointed Observations'  / Collection
DS_IDENT= 'ivo://cfa.harvard.edu/spec#10304' / Publisher DID for dataset
CR_IDENT= 'ivo://cfa.harvard.edu/tdc#MMT4302-102' / Creator internal ID for dataset
DATE    = '2004-08-30T14:18:17' / Date and time of file creation
VERSION =  2                     / Reprocessed 2004 Aug
TELESCOP= 'MMT '              / Telescope  [Not part of Spectrum DM]
INSTRUME= 'MMT/BCS '          / Instrument
FILTER  = 'G220    '          / Grating  [Not part of Spectrum DM]
CRETYPE = 'Archival'          / Not an on-the-fly dataset
VOLOGO  = 'http://cfa.harvard.edu/vo/cfalogo.jpg' / VO Creator logo
CONTRIB1= 'Jonathan McDowell'  / Contributor
CONTRIB2= 'Wilhelm Herschel'  / Contributor
CONTRIB3= 'Harlow Shapley'     / Contributor
DSSOURCE= 'Pointed'   /  Survey or pointed, etc
DER_SNR =         5.0  /   Estimate of signal-to-noise
DER_Z   =       0.01845   /  Redshift measured in this spectrum
DER_ZERR =       0.00010   /  Error in DER_Z
TIMESDIM= 'T'                  / Time SIDim 
SPECSDIM= '10-10 L'             / Spectral SIDim
FLUXSDIM= '10+7 ML-1T-3'       / Flux SDim
SYS_ERR =        0.05            / Fractional systematic error in flux
FLUX_CAL= 'Calibrated'   /
SPEC_ERR=        0.01 / Stat error in spec coord, in SPEC units
SPEC_SYE=        0.001   /   Frac sys error in spec coord
SPEC_CAL= 'Calibrated'
SPEC_RES=              5.0  / [angstrom] Spectral resolution
SPECBAND= 'Optical'    / SED.Bandpass
SPEC_RP =             800.0  / Spectral resolving power
SPEC_VAL=            4100.0  / [angstrom]  Characteristic spec coord
SPEC_BW =            1800.0  / [angstrom]  Width of spectrum
SPEC_FIL=               1.0  / No gaps between channels
TIME_CAL = 'Calibrated'   /
DATE-OBS= '2004-06-03T21:18:17' / Date and time of observation
EXPOSURE =   1500.015         / [s] Effective exposure time
TSTART   =  52984.301203      / [d] MJD
TSTOP    =  52984.318564      / [d] MJD
TMID     =  52984.309883     / [d] MJD mid expsoure
SKY_CAL  = 'Calibrated'  /
SKY_RES =              1.0  / [arcsec] Spatial.Resolution
RA      =     233.73791       / [deg] Pointing position
DEC     =      23.50333       / [deg] Pointing position
APERTURE=       2.0           / [arcsec]  Aperture diameter/Slit width
TIME     =  52984.309883      / [d] MJD of midpoint


COMMENT  ---------------------------
COMMENT  WCS Paper 3 Keywords
1S4_1    = 'WAVE'            / Column name with spectral coord
1CTYP4   = 'WAVE-TAB'        / Spectral coord is WAVE 
1S5_1    = 'WAVE'            / Column name with spectral coord
1CTYP5   = 'WAVE-TAB'        / Spectral coord is WAVE 
1S6_1    = 'WAVE'            / Column name with spectral coord
1CTYP6   = 'WAVE-TAB'        / Spectral coord is WAVE 
1S7_1    = 'WAVE'            / Column name with spectral coord
1CTYP7   = 'WAVE-TAB'        / Spectral coord is WAVE 
COMMENT  ---------------------------
TTYPE1 = 'WAVE'   / Wavelength
TFORM1 = '180E'
TUNIT1 = 'angstrom'
TUCD1  = 'em.wl'                    /
TDMIN1 = 3195.0 /
TDMAX1 = 5005.0 /
TUTYP1 = 'Spectrum.Data.SpectralAxis.Value'
TTYPE2 = 'WAVE_LO' /
TFORM2 = '180E'
TUNIT2 = 'angstrom'
TUTYP2 = 'Spectrum.Data.SpectralAxis.Accuracy.StatErrLow'
TTYPE3 = 'WAVE_HI' /
TFORM3 = '180E'
TUNIT3 = 'angstrom'
TUTYP3 = 'Spectrum.Data.SpectralAxis.Accuracy.StatErrHigh'
TTYPE4 = 'FLUX' /
TFORM4 = '180E'
TUNIT4 = 'erg cm**(-2) s**(-1) angstrom**(-1)'
TUTYP4 = 'Spectrum.Data.FluxAxis.Value'
TUCD4  = 'phot.fluDens;em.wl'     / Type of Y axis: F-lambda
TTYPE5 = 'ERR_LO' /
TFORM5 = '180E'
TUNIT5 = 'erg cm**(-2) s**(-1) angstrom**(-1)'
TUTYP5 = 'Spectrum.Data.FluxAxis.Accuracy.StatErrLow'
TTYPE6 = 'ERR_HI' /
TFORM6 = '180E'
TUNIT6 = 'erg cm**(-2) s**(-1) angstrom**(-1)'
TUTYP6 = 'Spectrum.Data.FluxAxis.Accuracy.StarErrHigh'
TTYPE7 = 'QUALITY' /
TFORM7 = '180I'
TUTYP7 = 'Spectrum.Data.FluxAxis.Quality'
\end{verbatim}
}

The data would look like
{\small
\begin{verbatim}

WAVE   WAVE_LO WAVE_HI FLUX  ERR_LO  ERR_HI  QUALITY
3200.0 3195.0 3205.0 1.48E-12 2.0E-14 2.0E-14   0
3210.0 3205.0 3215.0 1.52E-12 3.0E-14 3.0E-14   0
3220.0 3215.0 3225.0 0.38E-12 0.38E-12 0.0      0                  
3230.0 3225.0 3235.0 1.62E-12 3.0E-14 3.0E-14   0
...
5000.0 4995.0 5005.0 1.33E-11 3.0E-13 3.0E-13   1
\end{verbatim}
}

% \subsection{Alternate FITS serialization}
%
% Some implementors may find the one-row-per-segment approach inconvenient. An
% alternate `relational normalization' serialization, considered but
%currently rejected by the VO group, is recorded here for information. It
%would consist of two binary or ASCII tables, each with only scalar
%columns. The Spectrum  table would contain the same scalar columns as
%above, but the variable array columns would be moved to a second table,
%called SpecData, and used as scalar columns arranged vertically. A new
%integer column "OBS\_ID" would be added to each table; the values in the
%Spectrum table would be unique and the corresponding  values in the
%SpecData table would indicate which data points belonged to which rows
%in the Spectrum table.



